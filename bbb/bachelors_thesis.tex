\documentclass[]{scrartcl}

\setlength{\parindent}{0em}

%\usepackage[ngerman]{babel}
\usepackage[utf8]{inputenc}
\usepackage{csquotes}
\usepackage[OT2,T1]{fontenc}
\usepackage{amsthm}
\usepackage{amsmath}
\usepackage{amssymb}
\usepackage{mathtools}
\usepackage{nicefrac}
\usepackage{extarrows}
\usepackage[Algorithmus]{algorithm}
\usepackage{algorithmic}
\usepackage{caption}
\usepackage{subcaption}
\usepackage{tikz}
\usepackage{pgfplots}
\usepackage{graphicx}
\usepackage{color}
\usepackage{enumitem}
\usepackage{hyperref}
%\usepackage[german]{cleveref}
\usepackage{array}
\usepackage{marvosym}
\usepackage{tabularx}


\usepackage[
	backend=biber,
	style=alphabetic,
	sorting=ynt,
]{biblatex}
\addbibresource{thesis.bib}

\pgfplotsset{compat=1.16}

%Easier abs und norm
\DeclarePairedDelimiter\abs{\lvert}{\rvert}%
\DeclarePairedDelimiter\norm{\lVert}{\rVert}%
%
\makeatletter
\let\oldabs\abs
\def\abs{\@ifstar{\oldabs}{\oldabs*}}
%
\let\oldnorm\norm
\def\norm{\@ifstar{\oldnorm}{\oldnorm*}}
\makeatother

\DeclareMathOperator{\rank}{rank}
\DeclareMathOperator{\trace}{trace}

\newcommand{\R}{\mathbf{R}}
\newcommand{\newlinenice}{\newline \hspace*{5mm}}
\newcommand{\HRule}[1]{\rule{\linewidth}{#1}}
\newcommand{\hmu}{\hat{\mu}}
\newcommand{\hSig}{\hat{\Sigma}}
\newcommand{\bSig}{\bar{\Sigma}}
\newcommand{\tSig}{\tilde{\Sigma}}
\newcommand{\hSharpe}{\hat{\theta}^2}

\newtheorem{satz}{Satz}[section]
\newtheorem{theorem}[satz]{Theorem}
\newtheorem{lemma}[satz]{Lemma}
\newtheorem{kor}[satz]{Corollary}
\newtheorem{defi}[satz]{Definition}
\newtheorem{rem}[satz]{Remark}
\newtheorem{exa}[satz]{Example}



\begin{document}
	\begin{titlepage} % Suppresses displaying the page number on the title page and the subsequent page counts as page 1
	
	\raggedleft % Right align the title page
	
	\rule{1pt}{0.98\textheight} % Vertical line
	\hspace{0.05\textwidth} % Whitespace between the vertical line and title page text
	\parbox[b]{0.9\textwidth}{ % Paragraph box for holding the title page text, adjust the width to move the title page left or right on the page
		
		{\Huge\bfseries Implementierung eines\\[0.1\baselineskip]
		Neuroevolutionsalgorithmus zur\\[0.1\baselineskip]
		Optimierung von Topologie und Hyperparametern eines \\[0.1\baselineskip]
		künstlichen neuronalen Netzes} \\[0\baselineskip] % Titel
		
		{\Large Ein genetischer Algorithmus zur Automatiserung\\[0.2\baselineskip] von Trainingsprozessen. }\\[2.2\baselineskip] % Untertitel
		
		\vspace{0.1\textheight}

		{\huge Fachpraktikumsbericht}\\[2.4\baselineskip] % Subtitle

		{von \Large\textsc{justus will}}\\[2.4\baselineskip]
		
		\vspace{0.3\textheight}
		
		{\Large \today}
	}
	
\end{titlepage}

%This title page was originally created by Peter Wilson but has been extensively modified for 'https://www.latextemplates.com/template/vertical-line-title-page' by Vel.
	\thispagestyle{empty}
	\pagestyle{empty}
	\newpage
	\tableofcontents
	\newpage
	\pagestyle{plain}
	

\section{Einleitung}

	[On a stock market, a possible investor wants to maximize the amount of money he gains from buying assets and selling them at a later time. As we are not able to predict the behavior of the stock market, stochastic models are important for trying to solve this problem. We decide to model it by assuming a distribution that matches the returns of our assets, further using past data to estimate the specific parameters of it. If we are then able to fix the distribution, we can make the objectively best choice of assets we chose to buy and how many of them. Here objectively best choice means that given a specific utility function best describing our needs, our choice of assets will on average deliver the highest amount of utility. But as these parameters have to be estimated and the estimates depend on past data, the estimation has an included risk which we have to take into consideration. We call the  assets we hold a portfolio, and the method deciding what portfolio to buy a portfolio rule.
	
	For a long time portfolios were chosen by `proven in action' methods, which only use basic mathematical methods for their computation. One of them is the standard plug-in portfolio rule. As can easily be simulated, this rule scores poorly compared to other rules. In this thesis, we will study different plug-in rules in a mathematically accurate way and derive the `best' plug-in rule for different situations. Further we will prove that the often used standard plug-in rule is strictly dominated by other rules and consecutively is worse in every situation. To get these results, we expand the argumentation of Raymond Kan and Guofu Zhou in their paper 'Optimal Portfolio Choice with Parameter Uncertainty' \cite{kan_and_zhou}. We finish the thesis by simulating different portfolio rules on a dataset created by parameters of real stock market data, which will confirm the earlier results.
	] This indroduction is already done
\clearpage

	
\section{Problemstellung}

	Bevor wir uns genauer mit dem der Problemstellung der Optimierung von Topologie und Hyperparametern von Convolutional Neural Networks beschäftigen folgte zuerst eine kleine Einführung in neuronale Netze.
	Um besser verstehen zu können wofür neuronale Netze verwendet werden betrachten wir zunächst ein weitverbreitetes Standardproblem im Bereich des machinellen Lernens,
	dass wir später auch mit unserem Algorithmus lösen können:

	\subsection{Handschrifterkennung}\label{mnist}

		Eine Aufgabe für die sich neuronale Netzte hervorragend eignen ist Handschrifterkennung.
		Das Problem besteht darin eine Zahl die vorher noch nicht gesehen wurde nur anhand eines Bildes richtig zu klassifizieren, also auszugeben, was für eine Zahl abgebildet ist.
		Für Menschen ist diese Aufgabe mühelos lösbar, aber würde man ohne lernfähige Methoden einen Algorithmus zur Erkennung schreiben wollen, so wird dieser sehr kompliziert.
		Neuronale Netzte bieten hier eine simple und einfache Lösung.
		Der MNIST Datensatz \cite{mnist_data} besteht aus 70.000 Schwarz-Weiß-Bildern von Zahlen in einer Auflösung von 28x28 Pixeln. Die Aufgabe besteht darin mit 60.000 dieser Bildern (Trainingsdaten)
		Muster zu erkennen. Bewertet wird danach, wie gut der Algorithmus die 10.000 weiteren vom Algorithmus noch nicht gesehenen Zahlen (Testdaten) klassifiziert.

	\subsection{Klassifikationsprobleme}

		Die allgemeine Problemstellung die ein Neural Network löst ist die folgende:\\
		Gegeben sei eine Zielfunktion $f:\mathcal{H} \to \mathcal{G}$.\\
		Da diese Funktion aber nicht bekannt ist soll sie nun möglichst gut approximiert werden. \\
		Hierzu muss aus einer Menge von Funktionen $f_p:\mathcal{H} \to \mathcal{G}$ mit $p \in \Theta$ aus dem Parameterraum, diejenige ausgewählt werden, die am $f$ möglichst gut approximiert.
		In der Praxis ist es nicht leicht zu sagen, was es bedeutet, dass $f_p$ eine gute Approximation für $f$ ist.
		Für unsere Anwendung sind vorallem Klasifaktionsprobleme interessant, also (z.B. der Fall das $\mathcal{G}$ endlich ist und alle verschiedenen Klassen enthält.
		Um in diesem Fall die Güte quantifizieren zu können und um verschiedene Funktionen $f_p$ vergleichen zu können definieren wir die Genauigkeit auf den Testdaten,
		die aus dem Englischen \textit{classiication accuracy} auch kurz als \textit{Accuracy} bezeichnet wird:

		\begin{defi}[Classification accuracy] ~\\
			Zu der zu approximierenden Funktion $f$ betrachten wir eine Menge von $m$ Testdaten \\
			$(t_i, f(t_i)) \in \mathcal{G} \times \mathcal{H}$ $i \in \{1, \cdots , m\}$ mit korrekt klassifizierten Punkten $t_i$. \\
			Die Accuracy $acc$ ist nun definiert als:
			$$ acc(p) = \frac{1}{m} \sum_{k=1}^m \delta_{f(t_i)f_p(t_i)} \text{ wobei } \delta_{ij} =
			\begin{cases}
				1 & i = j \\
				0 & \, \text{sonst}
			\end{cases}$$
			Die Accuracy gibt also an welcher Anteil der Testdaten korrekt klassifiziert wird.
		\end{defi}
		
		Um nun geeignete Parameter aus $\Theta$ zu finden, ohne das wir die Testdaten verwenden dürfen, gibt es einen Trainingsdatensatz mit $n$ Daten\\
		$(x_i, f(x_i)) \in \mathbf{G} \times \mathbf{H}$ $i \in \{1, \cdots , n\}$, die ebenfalls bereits richitg klassifiziert sind.
		Aktuelle Methoden des machinellen Lernens optimieren nun die \textit{Accuracy} auf den Trainingsdaten und hoffen, dass dadurch auch die \textit{Accuracy} auf den
		Testdaten gut wird, also das neurale Netz gut generalisiert hat.\\

		Wenn wir erneut das Klasifaktionsproblem MNIST betrachten ergibt sich insgesamt z.B: \\
		$\mathcal{H} = \mathbb{R}^{28 \times 28}$, $\mathcal{G} = \{0, \cdots, 9\}$ sowie $n = 60.000$ Trainingsdaten und $m = 10.000$ Testdaten


	\subsection{Neuronale Netze}\label{nets}
		
		\subsubsection{Neuronen}

			Um verstehen zu können was neurale Netzte sind schauen wir uns zuerst den Grundbaustein an, aus denen Sie bestehen, die (künstlichen) Neuronen.
			Ein Neuron ist eine kleine Einheit die beliebig viele Inputs $x_1, \cdots, x_n$ erhält und daraus einen Output $y$ errechnet.
			Grafisch lässt sich das folgendermasßen vorstellen:

			% Neuron Bild

			\begin{defi}[Neuron] ~\\
				mathematisch gesehen handelt es sich bei einem Neuron um eine Funktion\\
				$y = g(\mathbf{x}) = \sigma(\sum_{j=1}^n \omega_jx_j + b)$ wobei $\sigma$ die Aktivierungsfunktion des Neurons ist.
			\end{defi}

			$\omega_j$ und $b$ sind dabei die Gewichte und der Bias des Neurons, diese beiden beinflussen, was genau das Neuron berechnen kann und müssen trainiert werden.
			Sie sind also Teil des Parameterraums $\Theta$ eines neuronalen Netzes.
			Sobald sie einmal festgelegt wurden ändern Sie sich jedoch nicht mehr, sind also unabhängig von $\mathbf{x}$.
			Wie $\sigma$ aussieht beeinflusst wie gut sich ein Netzt trainieren lässt und kann ausgewählt werden.
			Es handelt sich um den ersten Hyperparameter, der beeinflusst wie die Funktion $f_p$ für ein festes $p \in \Theta$ aussieht. Dazu später mehr.
			In der Praxis werden zwei Aktivierungsfunktionen häufig verwendet, Die Sigmoid- und die ReLU-Aktivierungsfunktion:

			\begin{defi}
				Die Sigmoidfunktion ist definiert als \\
				$$sig(z) \coloneqq \frac{1}{1+e^{-z}}$$
			\end{defi}
			\begin{defi}
				Eine weitere Funktion ist die ReLU-Funktion: \\
				$$ReLU(z) = \max(0, z)$$
			\end{defi}

			Beide Funktionen haben Eigenschaften die sie zu guten Kandidaten für Aktivierungsfunktionen machen, da Sie das schnelle und effektive Trainieren von Neuronalen Netzten ermöglichen.

		\subsubsection{Fully Connected Layer}

			Ein neuronales Netzt besteht nun aus sogenannten \textit{Layern} von Neuronen. Das sind Schichten die viele Neuronen enthalten, die nicht untereinander, aber mit den
			Neuronen der benachbarten Schichten, bzw \textit{Layern} verbunden sind. \\
			Eine \textit{Layer} fasst also die einzelnen Funktionen der $m$ Neuronen $g_i$ zu einer großen Funktion $g$ zusammen.
			$g$ operiert auf einem Vektor $\mathbf{X}$, der als Komponenten alle Ouputs der vorherigen Layer enthält.
			Jedes $g_i$ enthält immernoch seine eigenen Parameter $\omega_1^i, \cdots, \omega_n^i$ und $b^i$.
			
			In normalen neuronalen Netzten gibt es nur sogenannte \textit{Fully Connected Layer}, das sind \textit{Layer}, bei denen der Output jedes Neurons
			einer der Inputs jedes Neurons in der nächsten \textit{Layer} ist. Die beiden \textit{Layer} sind also vollständig miteinander verbunden.
			Eine \textit{Fully Connected Layer} mit $m$ Neuronen mit Funktionen $g_1, \cdots, g_m$ hat also die oben angesprochene Funktion\\
			$g(\mathbf{x}) = (g_1(\mathbf{x}), \cdots, g_m(\mathbf{x}))^T$

			Der Input der ersten Layer von Neutronen ist $\mathbf{x} \in \mathcal{H}$. Die einzelnen Komponenten von $\mathbf{x}$ bilden ebenfalls eine \textit{Layer},
			die sogeannte \textit{Input Layer}. Die letze \textit{Layer} aus Neuronen heißt auch \textit{Output Layer}, da ihr Output die Komponenten von $f_p(\mathbf{x}) \in \mathcal{G}$ darstellt.
			Jede anderen \textit{Layer} von Neuronen heißt \textit{hidden (versteckte) Layer}. Schon mit nur einer \textit{hidden Layer} kann man jede stetige Funktion
			z.b. bzgl. $\norm{\cdot}_{sup}$ beliebig gut approximieren. Dieses \textit{Universalitätstheorem} erklärt warum sich neuronale Netze in so unterschiedlichen Problemen anwenden lassen.

			Im Fall unseres MNIST Datensatzes wäre z.B. ein einfaches Netzwerk mit einer \textit{hidden Layer} denkbar:

			% Abbild Layer
			
		\subsubsection{Convolutional Layer}

			Da sich das Hauptaugenmerk unsere Algorithmus auf Klasifaktionsproblemen in der Bilderkennung liegt, betrachten wir zudem die dort üblichen \textit{Convolutional Neural Networks}.
			Dises erweitern die Idee der neuronale Netze um eine weitere Art zwei \textit{Layer} zu verbinden - die Faltung, im Englischen \textit{Convolution}.
			Anders als bei den vollständig verbunden \textit{Layern} werden nur die räumlich lokalen Nachbarn zusammengefasst. Das heißt der Output eines Neurons ist nicht für jedes Neuron der nächsten
			\textit{Layer} ein Input, sondern nur für manche (räumlich Nahe). Außerdem hat nicht jedes Neuron seine eigenen Gewichte $\omega_j$ sondern es gibt einen Faltungskern (\textit{Kernel})
			der bestimmt wie die einzelnen Inputs gewichtet werden.
			Wir betrachten nur die zweidimensionale Faltung, dafür müssen die einzelnen \textit{Layer} nicht eindimensional wie in Abildung %2
			sondern zweidimensional angeordnet werden. \\
			Sei $\mathbf{X} \in \R^{m \times n}$ der Input der \textit{Layer}. Im Gegensatz zu einer \textit{Fully Connected Layer} wo\\
			$$g(\mathbf{X})_{ij} = g_{ij}(\mathbf{X}) = \sigma(\sum_{k_1=1}^m \sum_{k_2=1}^n \omega_{k_1k_2}^{ij}\mathbf{X}_{k_1k_2} + b^{ij})$$
			gelten würde, gibt es nun einen Faltungskern $K \in \R^{\hat{m} \times \hat{n}}$ und es gilt
			$$g(\mathbf{X})_{ij} = g_{ij}(\mathbf{X}) = \sigma(\sum_{k_1=0}^{\hat{m} - 1} \sum_{k_2=0}^{\hat{n} - 1} K_{k_1k_2}^{ij} \mathbf{X}_{i + k_1, j + k_2} )$$
			Oft wird zudem $\sigma(z) \coloneqq z$ gewählt, bzw. kein $\sigma$ verwendet.
			Intuitiv lässt sich die Faltung so interpretieren, dass der Faltungskern über den Input läuft und an jeder Stelle einen Ouptut generiert. Dieser Vorgang ist hier noch einmal dargestellt:

			%Abbild Convo

			Es fällt zudem auf, dass die Größe von $g(\mathbf{X})$ nun nicht mehr beliebig gewählt werden kann, wie bei der \textit{Fully Connected Layer}, stattdessen ist die Größe durch die Faltung
			eindeutig bestimmt. Es gilt $g(\mathbf{X}) \in \R^{m' \times n'}$ mit $m' = m - \hat{m} + 1$ und $n' = n - \hat{n} + 1$.

			Die Gewichte $K$ des Faltungskerns werden trainiert und sind Teil des Parameterraums $\Theta$ während $\hat{m}$ und $\hat{n}$ Hyperparameter sind.

		\subsubsection{Pooling Layer}
			In \textit{Convolutional Neural Networks} gibt es noch eine zweite neue Art \textit{Layer} zu verbinden, die \textit{Pooling Layer}.
			Sie funktionieren sehr ähnlich zu der \textit{Convolutional Layer} gibt es aber keinen Faltungskern mit Gewichten, sondern die räumlich nahen Inputs werden mit einer anderen
			einfachen Funktion verknüpft:
			\begin{align*}
				g&(\mathbf{X})_{ij} = g_{ij}(\mathbf{X}) \\
				= p&ool(\mathbf{X}_{i, j}, \cdots, \mathbf{X}_{i, j + \hat{n} - 1}, \mathbf{X}_{i + 1, j}, \cdots, \mathbf{X}_{i + 1, j + \hat{n} - 1},
				   \cdots, \mathbf{X}_{i + \hat{m} - 1, j}, \cdots, \mathbf{X}_{i + \hat{m} - 1, j + \hat{n} - 1})
			\end{align*}
			Dies ist meistens das Maximum oder der Durschnitt. Man spricht von \textit{Max Pooling Layer} oder \textit{Average Pooling Layer}.

	\subsection{Parameter}\label{train}
		Wir haben nun gesehen, das neuronale Netze eine Funktion $f_p$ darstellen, die von Gewichten und bias, also den Parametern $p \in \Theta$ abhängt.
		Diese Parameter sind die trainierbaren Parameter des neuronalen Netzes und unterscheiden sich so von den nicht trainierbaren Hyperparametern.
		Das Ziel besteht nun $p$ möglichst gut bezüglich der Testgenauigkeit (\textit{Accuracy}) zu wählen. Das geschieht in dem Lernverfahren von gewissen Anfangsparametern $p_0$
		iterativ verbessert werden. Diesen iterative Vorgang nennt man "trainieren". \\
		Dazu werden auf dem Gradientenabstieg (\textit{Gradiant Descent}) basierende Methdoden verwenden, die eine Kostenfunktion minimieren,
		die angibt wie gut $f_p$ die Trainingsdaten vorhersagen kann. Für das Lernen bieten aktuelle Bibliotheken bereits vorgefertigte Optimierer, wie z.B.
		in \textit{PyTorch} den \textit{Stochastic gradiant descent} (\textit{SGD}) oder den Optimizier \textit{ADAM}.
		Alle Optimizer müssen mit verschiedenen Hyperparametern wie der \textit{learning rate} eingestellt werden, die z.b. angibt wie schnell/fein sich die Werte verbessern. \\
		Ohne zu viel auf diese Hyperparameter eingehen zu wollen ist klar, das die Auswahl dieser Hyperparameter ein gewisses Verständnis des Optimierers voraussetzt und einen
		großen Einfluss auf die Trainingsgeschwindigkeit und Güte des Ergebnisses hat. \\

		Zusätzlich gibt es noch weitere nicht trainierbare Hyperparameter, darunter unter anderem die Anzahl und Art von \textit{Layern} und welche \textit{Layer} untereinander verbunden sind.
		Man spricht hier oft von der Topologie des Netzes. Des weiteren gibt es noch die Hyperparameter der einzelnen Layern wie
		die Anzahl der Neuronen (bei einer \textit{Fully Connected Layer}) oder die Größe des Faltungskerns (bei einer \textit{Convolutional Layer}).
	
	\subsection{Hyperparametersuche}
		Insgesamt gibt es also jede Menge Hyperparameter, die gewählt werden müssen und ein Vielzahl von möglichen Topologien. Ohne gründliches Wissen über neuronale Netze
		ist es für den Einsteiger sehr schwer geignete Hyperparamter zu finden. In der Praxis hat die Erfahrung gezeigt, dass manche Dinge bessern funktionieren als andere.
		Mit genügend Erfahrung kann man strukturiert verschiedene Hyperparameter austesten. Diese durch Erfahrung gewonnen Erkenntnisse finden sind zum Beispiel in dem Paper
		"Practical Recommendations for Gradient-Based Training of Deep Architectures" von Yoshua Bengio \cite{parameters}.

		Einem Anwender, der sich mit neuronale Netzen nicht gut auskennt, sollte es aber im besten Fall erspart werden, sich so tief in die Materie einlesen zu müssen.
		Obwohl moderne Bibliotheken wie \textit{PyTorch} mit bereits Implementierten Methoden und Klassem viele lästige und komplizierte Aufgaben bereits abnehmen,
		und so auch Nicht-Experten das Experimentieren mit neuronalen Netzen ermöglichen wäre es wünschenswert auch die Wahl der Hyperparameter zu automatisieren.\\

		Das Ziel ist es, dass ein Anwender nur deklarativ die Trainings und Testdaten angibt und ein Algorithmus automatisch das passende Netz auswählt und
		es trainiert, so dass mit genügend Rechenzeit auch ohne Expertise ein gutes Resultat ensteht.
		Auch wenn man schon alles über neuronale Netze weiß ist dies trotzdem erstrebenswert, da so mehr Zeit für wichtigere Aufgaben bleibt,
		während die immer besser werdenden Computer die Rechenarbeit übernehmen

	\subsection{bisherige Ansätze}\label{ansatz}

		Aufgrund der Bedeutendheit der Hyperparametersuche ist es nicht verwunderlich, dass schon viele Anstrengungen in diese Richtung unternommen werden.
		Wir betrachten nun ein paar klassische Ansätze der Hyperparametersuche, wie sie auch in \cite{parameters} beschrieben werden.
		Allen Ansätzen ist es gleich, das viele Netze nacheinander trainiert werden müssen. Danach wählt man das beste aus allen betrachtenden Netzen.

		\subsubsection{Grid Search}\label{grid}

			Für jeden Parameter der gewählt werden soll kan man ein Intervall angeben, in dem nach dem Parameter gesucht werden soll.
			So kann man den Suchraum definieren, in dem man nach den besten Hyperparametern suchen möchte.
			
			Die Idee des \textit{Grid Search} ist es nun für jedes dieser Werteintervalle ein paar Werte auszuwählen.
			Für jede Kombination aus möglichen Werten, lässt man nun einmal das entstehende Netz trainieren, bis schlussendlich die besten Hyperparameter gefunden wurden.

			Durch geschicktes Design kann \textit{Grid Search} noch verbessert werden, z.B. indem man bei der Auswahl der Werte aus dem Werteintervall
			Wissen einfließen lässt, wie z.B. das sich die \textit{learning rate} des Optimizers für ähnliche Größenordnungen auch ähnlich verhält und 
			man deshalb besser logarithmisch linear Werte auswählt also z.B. [0.1, 0.01, 0.001, 0.0001].
			Auch kompliziertere Erweiterungen wie geschatelte Suchen mit immer höherer Auflösung oder
			nur ein paar Hyperparameter auf einmal zu testen und dafür mehrere Tests durchzuführen kann die Güte und Geschwindigkeit des Verfahrens verbessern.
			Für weitere Details siehe \cite{parameters}.

			Trotzdem bleibt \textit{Grid Search} sehr rechenaufwändig, da die Anzahl an Tests exponentiell in der Anzahl der Hyperparameter ist.


		\subsubsection{Random Search}

			Im Gegensatz zur \textit{Grid Search} die systematisch den Suchraum absucht können durch zufällige Wahl von Hyperparametern erstaunlicherweise
			viel schneller und soagar bessere Ergebnisse erzielt werden. \cite{randomsearch}
			Für jeden Parameter wird eine Verteilung angeben, meistens eine Gleichverteilung über das logaritmische Werteintervall (siehe \ref{grid})
			oder eine multinomiale Verteilung bei diskreten Hyperparametern.

		\subsection{NEAT}

			Einen ganz anderen Ansatz verfolgen sogenannte genetische Algorithmen. Besonders hervorzuheben ist hier das Paper
			"Evolving Neural Networks through Augmenting Topologies" \cite{neat} aus dem Jahr 2002, das den Grundstein für alle vergleichbaren Methoden gelegt hat.
			Die Idee ist simpel. Um eine bestmögliche Netztopologie zu finden, kann man mit einem möglichst kleinen Netz anfangen und durch Mutation und Crossover
			neue immer größere Netze erzeugen, die immer besser werden. Sobald die enstehenden Netzt nicht mehr besser werden kann man aufhören.

			Ein Nachteil an NEAT ist aber, dass der Algortihmus in einer Zeit entwickelt wurde, als die Computer noch nicht so viele Möglichkeiten hatten wie heutzutage
			und außerdem zeitdem viele Fortschritte im Bereich des machinellen Lernens gemacht wurden, die bei \textit{NEAT} nicht berücksichtigt werden konnten.
			In \textit{NEAT} werden einzelne Verbindungen und Neuronen erzeugt und mutiert. \textit{NEAT} ist nicht darauf ausgelegt moderne Netze
			mit mehreren Tausend Neuronen zu erzeugen, sondern nur höchstens ein paar Hundert. Das ist für aktuell Zwecke nicht mehr ausreichend.
			Außerdem sieht \textit{NEAT} vor das auch die Parameter in $\Theta$ durch Evolution trainiert werden. Hierfür stehen
			mittlerweile viel bessere und performantere Methoden wie \textit{SGD} oder \textit{ADAM} zur Verfügung. \ref{train}

	\clearpage

	\section{convNEAT}

		\textit{NEAT} liefert uns die Grundidee für unseren Algorithmus. \\
		Wir übertragen das Konzept in die Moderne, in dem wir nicht mit einzelnen Neuron arbeiten,
		sondern als Grundeinheit direkt ganze \textit{Layer} von Neuronen betrachten und auf und zwischen diesen Mutationsoperationen definieren.
		Das Training auf den Trainingsdaten überlassen wir einem moderneren Optimierer wie in \ref{train}.
		Zusätzlich betrachten wir nicht nur neurale Netze sondern auch \textit{Convolutional Neural Networks} die unserem Algorithmus den Namen \textit{convNEAT}
		verleihen. So können wir besonders Klasifaktionsprobleme in der Bildverarbeitung wie den MNIST Datensatz \ref{mnist} besser lösen.

		Es gibt bereits ähnliche Versuche die Ideen von \textit{NEAT} auf \textit{Convolutional Neural Networks} wie etwa
		\textit{EXACT} von T. Desell \cite{exact} oder einen Ansatz von Y. Sun et al \cite{convoneat}.
		Beide Ansätze haben aber Nachteile gegenüber \textit{convNEAT} zeigen aber das ein genetischer Ansatz durchaus zielführend sein kann.
		Sie verwenden keine Kapselung der Netze in Species wie bei \textit{NEAT} und erlauben beide nur begrenzte \textit{Convolutional Neural Networks}.
		ConvNEAT bietet eine größere Flexibilität und mehr Möglichkeiten für beliebige \textit{feedforward Convolutional Neural Networks}
		auch mit \textit{Pooling Layern} sowie leichter Erweiterbarkeit und Anpassbarkeit. Durch modernes \textit{Clustering} können
		bessere Ergebnisse erzielt werden und durch die Evolution von allen Hyperparametern inklusive den Hyperparametern der Optimierers
		werden dem Anwender alle Entscheidungen abgenommen.
		
		Bevor wir auf die genaueren Details von convNEAT eingehen, werden wir zuerst ein paar andere Grundlegende Probleme ansprechen.

		\subsection{Representation}

			Da wir mit einem evolutionären Algorithmus arbeiten, müssen wir eine geieignete genetische Representation finden.
			Eine Kodierung der Gene eines Netzes in ein Genom bestimmt wie das Netz aussieht und muss deshalb alle Hyperparameter enthält, die uns interessieren.
			Neben der Anzahl der Größe und \textit{Layer} so wie deren Hyperparameter gehören aber auch alle anderen Hyperparameter z.B. auch
			die des Optimieres dazu.

			Die Representation bestimmt schlussendlich welche Netze gebildet werden können, also auch den Suchraum in dem gesucht werden muss.
			Dieser Suchraum kann dank des evolutionären Ansatzes viel größer sein als bei einer \textit{Grid Search} oder \textit{Random Search} \ref{ansatz}.
			Y Sun et al \cite{convoneat} verwenden z.B. Listen variabler Länge um Netze darzustellen.
			Der Suchraum wird beschränkt auf diejenigen \textit{Convolutional Neural Networks}, die erst eine Reihe \textit{Convolutional Layer}
			und danach eine Reihe von \textit{Fully Connected Layer} aufweisen.
			Bei convNEAT wollten wir unseren Suchraum nicht so weit einschränken, sondern alle möglichen \textit{feedforward Convolutional Neural Networks} durchsuchen.

			convNEAT verwendet deshalb eine Kodierung die an die Grundidee von \textit{NEAT} angelehnt ist.
			\textit{NEAT} kodiert neuronale Netze als Graph, wobei die Neuronen die Knoten und die Verbindungen mit Gewichten die Kanten sind.
			Natürlich muss dieser Ansatz angepasst werden, trotzdem kann man sich jedes \textit{Convolutional Neural Net}
			also auch jedes gewöhnliche neuronale Netz als einen Graphen vorstellen:
			Die Knoten sind die einzelnen \textit{Layer} und die Verbindungen zwischen den einzelnen \textit{Layern} werden über die Kanten beschreiben.
			Wie wir bereits in \ref{nets} gesehen haben hängt das davon ab um was es sich bei der hinteren \textit{Layer} handelt.
			Diese Information, ob \textit{Fully Connected Layer}, \textit{Convolutional Layer} oder \textit{Pooling Layer} ist zusammen mit allen
			zugehörigen Hyperparametern in den Kanten gespeichert.

			Wie bereits erwähnt ist die Anzahl der Neuronen in den Knoten nicht immer frei wählbar, sondern nur falls es sich, bei der eingehenden Kante
			um eine \textit{Fully Connected Layer} handelt. Aus diesem Grund wird die Größe nicht in den Knoten kodiert,
			stattdessen wird die Größe durch das Netz propagiert und in jeder Kante bearbeitet. Für \textit{Fully Connected Layer} wird nur gespeichert
			wie groß die Änderung der Größe ist, also zum Beispiel, dass die \textit{Fully Connected Layer} 20 mehr Neuronen enthält als der Vorgänger.
			Es können auch inaktive Kantengene in Genom gespeichert werden.

			Wenn wir beliebige gerichtete Graphen zulassen stoßen wir auf zwei Probleme:
			Es kann zu Kreisen kommen, so dass das Durchpropagieren des Inputs nicht mehr funktioniert. \\
			Diesem Problem können wir entgegenwirken, in dem wir beim Hinzufügen einer Kante darauf achten, dass keine Kreise enstehen. \\
			Außerdem ist es möglich das ein Knoten zwei eingehende Kanten besitzt, die nicht zueinander passen.
			In diesem Fall müssen durch Hochskalierung, Runterskalierung, einer Mischung aus beiden oder durch das Hinzufügen von Nullen die 
			Ouputs auf die selbe Größe und konkatiniert werden. Wie genau das passiert ist ein weiterer Hyperparameter der im Knoten kodiert wird.

			Für maximale Flexibilität besonders auf die Anwendung der Bildanalyse bezogen, sind alle \textit{Layer} im späteren Netz, dreidimensional.
			Neben Höhe und Breite gibt es noch verschiedende Channels, in denen z.B. im Input Gelb, Rot und Blau kodiiert werden könnten.
			Bei MNIST bleibt der Input dann z.B. $\mathbf{X} \in \mathbb{R}^{1 \times 28 \times 28}$.
			Das sorgt dafür, dass im Verleich zu z.B. EXACT \cite{exact} viel weniger Knoten gebraucht werden, weil viel mehr Faltungen kompakt
			representiert werden können.

			In Abb. finden sich zwei Beispiel für Netze und ihre Representation.

			%Abbild

			Neben der Topologie wird auch noch kodiert welcher Optimierer gerade mit welchen Hyperparametern verwendet wird.

		\subsection{Mutation}

			Die Möglichkeit Genome zu mutieren bildet die Basis jedes genetischen Algorithmuses, so können neue Netze erzeugt werden, die den
			bisherigen Netzen ähneln. Schlechte Mutationen können in der Selektion aussortiert werden, gute Mutationen bleiben erhalten.
			Welche Mutation passiert ist zufällig, manche haben jedoch höhere Wahrscheinlichkeit als andere.
			Die Wahrscheinlihckieten wurden durch viele Testläufe so angepasst, dass Sie sinnvoll sind.
			Diese Hyper-Hyperparameter müssen jedoch nur einmal bestimmt werden und sind Porblemunabhängig.
			Trotzdem wäre es denkbar auch eine adaptive Veränderung der Wahrscheinlichkieten einzuabeun, um die Mutationen häufiger auftretten zu lassen,
			die häufiger zu guten Netzen führen.

			\textit{convNEAT} bietet folgende Mutationen:

			\begin{enumerate}
				\item [] \textbf{Aktivieren und Deaktivieren von Genen} \\
					Es gibt die Möglichkeit Gene zu deaktivieren und wieder zu reaktivieren. Deaktivierte Gene spielen keine Rolle mehr für das entstehende Netz.
					Deaktivierte Gene können auch durch den \textit{Crossover} entstehen. Beim Deaktivieren muss sichergestellt sein, dass
					es noch einen Pfad vom Input zum Output gibt, damit noch ein gültiges Netz entsteht.

				\item [] \textbf{Kante aufteilen} \\
					Es gibt die Möglichkeit eine bestehende Kante aufzuteilen. Dabei entsteht ein neuer Knoten zwischen zwei existierenden Knoten.
					Die ursprüngliche Verbindungskante wird deaktiviert und zwei neue Kanten eingefügt. Die erste ist eine Kopie der alten Kante.
					Die zweite neue Kante ist eines zufälligen Typs. Manche Kanten sind hier wahrscheinlicher, z.B. ist hiner einer
					\textit{Convolutional Layer} eine neue \textit{Pooling Layer} am wahrscheinlichsten.

				\item [] \textbf{Kante einfügen} \\
					Zwischen zwei Knoten kann eine Kante eingefügt werden. Wie beim Aufteilen ist die Art der Kante zufällig.
					Damit keine Kreise entstehen können, wird für jeden Knoten seine Tiefe im Netz gespeichert.
					Eine neue Kante zeigt dann immer auf die tiefere Kante, beliebige Pfade durchlaufen dann immer Knoten in echt absteigender Reihenfolge,
					Kreise sind unmöglich.

				\item [] \textbf{Optimierer} \\
					Welcher Optimierer verwendet wird und alle seine Hyperparameter können ebenfalls mutiert werden.
					Dies kann unter anderem die \textit{learning rate} oder der \textit{weight decay} sein.

				\item [] \textbf{Kante mutieren} \\
					Alle Kantentypen können ihre Hyperparameter mutieren, jede Mutation hat seine typspezifische Wahrscheinlichkeit.
					Unter anderem sind folgende Mutationen möglich:
					\begin{enumerate}
						\item [] \textbf{Fully Connected Layer} \\
							Die Aktivierungsfunktion und die Änderung der Größe können mutieren.
						\item [] \textbf{Convolutional Layer doer Pooling Layer} \\
							Die Weite und Höhe und Tiefe des Kernels sowie weitere Parameter wie unter anderem \textit{padding} oder \textit{stride} können mutieren.
					\end{enumerate}
			\end{enumerate}

		\subsection{Selektion}\label{select}
			
			Wichtig für jeden genetischen Algorithmus ist der Operator der \textit{Selektion}. Die besten Netze werden beibehalten und die schlechtesten,
			z.B. diejenigen die durch eine unvorteilhafte Mutation entstanden sind sollten nicht weiter überleben.
			Ein gutes Maß für die Güte eines Netzes ist die \textit{Accuracy} auf den Testdaten. Da beim öfteren Vergleich der \textit{Accuracy} auf den Testdaten,
			aber die Gefahr des \textit{Overfittings} besteht, also die Chance, dass das Netz auf neuen Daten keine guten Vorhersagen macht, nicht mehr gut generalisiert,
			wird wie es üblich ist ein Teil der Trainingsdaten nicht zum Trainieren verwendet sondern als Validierungsdaten zurückgehalten.
			Bei der Selektion kann dann die \textit{Accuracy} auf den Validierungsdaten bestimmt und verglichen werden.
			Die Testdaten werden nur ganz am Ende verwendet um die Güte des finalen Netzes aus \textit{convNEAT} zu überprüfen.\\

			Damit Netze die schon länger trainiert haben keinen Vorteil gegenüber frisch mutierten Netzen haben gibt es für jede trainierte Epoche eine Strafe.
			Die Strafe ist linear im logarithmischen Klassifikationsfehler, das heißt:

			\begin{defi}[Log classification error] ~ \\
				Der logarithmische Klassifikationsfehler eines Netzes mit Accuracy $a$ ist definiert als \\
				$$ logerr(a) = \log_{10}(1 - a)$$
			\end{defi}
		
			Für ein Genome mit Accuracy $a$ und $n$ trainierten Epochen kann man einen modifizierten score berechnen mittels\\
			$$score(a, p) =  1 - 10^{logerr(a) + n * decay}$$
			Denn so gilt:
			$$ logerr(score(a, p)) = logerr(a) + n * decay$$

			Basierend auf diesem angepasstem score können jetzt Paare aus Genomen gebildet werden die im nächsten Schritt, dem \textit{Crossover} kombiniert werden können.
			Es gibt mehrere Möglichkeiten diese Selektion durchzuführen. In \textit{convNEAT} gibt es mehrere Möglichkeiten:
			Von einfachen Methoden, wie die Auswahl aller besten Genome, oder der zufällifen Wahl, bei der bessere Gene höhere Wahrscheinlichkeiten,
			die von dem Score abhängen sind haben, bis hin zu komplizierteren Methoden wie der \textit{Tournament selection} bei der zufällig eine
			Gruppe von $k$ Genomen ausgewählt wird von denen das Beste ausgewählt wird oder \textit{stochastic universal sampling} \cite{sus}.

			Welche Methode die beste ist, muss mit viel Testläufen auf verschiedenen Problemen ausgetestet werden, da meine Rechenzeit aber limitiert war konnte
			ich nur einige wenige Tests auf dem MNIST Datensatz durchführen. Hier schien es, als ob \textit{stochastic universal sampling}
			eine stabile und effektive Methode der Selektion ist.

			Wichtig ist das die Selektion auch ein paar schlechtere Netze überleben lässt. Denn vielleicht sind manche Netze nicht lange genug trainiert worden
			und zeigen ihr volles Potenzial nach etwas mehr Zeit. \textit{stochastic universal sampling} besitzt diese Eigenschaft.

			Um eine Intuition für die verschiedenen Selektionsfunktion zu bekommen sind sie hier verglichen:

			% Abbild Selektion
		
		\subsection{Crossover}

			Eine wetere Operation in genetischen Algorithmen ist der \textit{Crossover}. Er erzeugt aus zwei Genomen, ein neues Genom.
			Dieses neue Genome erhält alle Gene der Eltern. Wichtig für \textit{convNEAT} sind, wie bei \textit{EXACT} \cite{exact},
			die Parameter \textit{more fit parent crossover rate} und \textit{less fit parent crossover rate}.
			Diese beiden beschreiben wie viel Prozent des jeweiligen Genomes aktiviert bleiben soll.
			Gene (hier vor allem die Kanten), die beispielsweise nur im besseren Genom (mit höherem Score \ref{select}) vorkommen,
			werden mit Wahrscheinlichkeit \textit{more fit parent crossover rate} aktiviert übernommen.
			Alle anderen Gene werden deaktiviert insofern das möglich ist.

			Da die Eltern zwei beliebige Graphen sein können, ist es schwierig zu sagen, welche Gene gleich sind und wie die nicht gleichen Gene kombiniert werden sollen.
			Abhilfe schafft das Konzept der \textit{historical markers} aus \textit{NEAT}. Jedes Gene bekommt eine eindeutige Zahl zugeordnet und wenn durch
			Muatation neue Kanten oder Knoten entstehen werden ihnen neue Zahlen zugeordnet.
			Möchte man zwei Genome vergleichen, kann man sich die \textit{Marker} anschauen und sieht sofort wie viele Genome gleich sind und wie
			unterschiedlich (im biologischen Sinne) die Genome sind.

			Ein simples Beispiel für einen Crossover findet sich in in Abb. 

			% Abbild Crossover

		\subsection{Clustering}

			Mit den obigen genetischen Operatoren kann ein genetischer Algorithmus entworfen werden, der theoretisch immer bessere Netze erzeugen kann.
			In der Praxis tritt man jedoch auf ein Problem das schon bei der Entwicklung von \textit{NEAT} bekannt war.
			Wird ein neues Netz generiert, z.B. durch \textit{Crossover} so kann es in der Praxis oft nicht lange überleben, bis es durch die Selektion
			aussortiert wird. Das neue Netz braucht aber einige Zeit um durch Training, oder weitere Mutationen langsam besser zu werden, bis es
			kompetetiv genug ist, um mit den aktuell besten Netzen verglichen werden kann, die schon über viele Generationen optimiert wurden.

			Eine Lösung für das Problem ist wie bei Y. Sun et al \cite{convoneat} eine Selektion auszuwählen, die nur sehr geringen
			\textit{Selektionsdruck} ausübt, also auch schwache Netze oft überleben lässt. Da aber mit Senkung des \textit{Selektionsdruck}
			auch die Geschwindigkeit sinkt, mit der Fortschritte gemacht werden, verwendet \textit{convNEAT} zusätzlich das Konzept
			der \textit{Artentrenung} (im Englischen \textit{Speciation}), im weiteren auch mit \textit{Clustering} bezeichnet.

			Ähnliche Netze mit ähnlicher Topologie werden in einzelne \textit{Arten} (\textit{Cluster}) aufgeteilt, so dass Netze im gleichen
			Cluster möglichst ähnlich und in verschiedenen Cluster möglichst verschieden sind.
			Die Selektion erfolgt jetzt nur in den einzelnen Clustern, so dass es dauerhaft Cluster von Netzen geben kann, die sich von dem besten Cluster unterscheiden.
			Während das beste Cluster weiter optimiert wird gibt \textit{convNEAT} auch anderen Ansätzen die Chance erforscht zu werden und optimiert zu werden.

			Das Zusammenspiel der einzelnen Cluster untereinander muss auch genau geregelt sein. \textit{convNEAT} sorgt dafür, dass
			die besten Cluster größer werden können, um mehr Rechenzeit in die Optimierung der aktuellen Lösung zu investieren, aber gleichzeitig, andere
			Cluster nicht zu klein werden, auch wenn sie gerade schlechter sind. Cluster die sich nicht weiter verbessern, weil seine Netze einfach
			nicht gut für das Problem sind werden irgendwann gelöscht.

			Damit \textit{Clustering} funktionieren kann brauchen wir zwei Dinge: Eine Abstandsfunktion und einen Clusteralgorithmus.

			\subsubsection{Ähnlichkeit}

				Das Konzept von \textit{"ähnlichen"} Genomen muss nun mathematisch definiert werden.
				Für jede Art von Gen kann ein Ahnlichkeit zwischen $0$ und $1$ defineirt werden, in Abhängigkeit davon wie unterschiedlich
				die Hyperparameter sind. $0$ bedeutet, dass die verglichenen Gene gleich sind
				und je unterschiedlicher zwei Gene sind, desto näher wird die Ähnlichkeit zur $1$.

				\begin{defi} [Abstandsfunktion]
					\textit{convNEAT} verwendet folgende Funktion: \\
					$$dist = c_0*S + c_1*D + c_2*E + c_3*T + c_4*K + c_5*X$$
					wobei $S$ der Unterschied in den gemeinsamen Kanten ist, \\
					$D$ und $E$ zusammen die Anzahl der Gene die nur in einem der beiden Genome vorkommt, \\
					$T$ ist der Unterschied der verwendeten Optimierer, $K$ der Unterschied in den Knoten und $X$ der Unterschied in der Anzahl der trainierten Epochen.
				\end{defi}
				
				Diese Abstandsfunktion sorgt dafür, dass alle Unterschiede eine festlegbare Rolle spielen.
				Die Parameter $c$ wurden durch Testläufe optimiert. Mit Abstand am wichtigsten sind $S$, $D$ und $E$.
				Trotzdem bleibt das Konzept von Ähnlichkeit leicht subjektiv.

			\subsubsection{Clusteralgorithmus}
				
				Das Problem eine Menge von Daten $\mathbf{X}$ in $k$ Cluster einzuteilen lässt sich als Optimierungsproblem darstellen. Zu minimieren ist der
				Abstand der Mitglieder der Clusters $S_i$ zum jeweiligen Clustermittelpunkt $\mu_i$:

				\begin{defi}
					$$J \coloneqq \sum _{i=1}^{k}\sum _{x_j \in S_i} dist(x_j, \mu_i)^2$$
				\end{defi}

				Für den Fall $\mathbf{X} \in \mathbb{R}^m$ gibt es hervorragende Algorithmen wie \textit{k-Means} \cite{kmeans}, um das Optimierungsproblem
				approximativ und sehr schnell zu lösen. \textit{k-Means} ist ein iteratives Verfahren, dass ausgehen von Start-Clustermittelpunkten $\mu_i^0$
				iterativ immer bessere $\mu_i$ findet, in dem es den Mittelpunkt der aktuellen zu Cluster $S_i$ gehörenden Datenpunkt zum neuen $\mu_i$ macht
				bis dieser Prozess konvergiert.

				Leider lässt sich \textit{k-Means} nicht auf unser Problem anweden, da der Suchraum kein Vektorraum mit einer Basis ist,
				der Begriff eines Mittelpunktes ist nicht defniert. Wir müssen lediglich mit dem Abstand $dist$ arbeiten. 

				\textit{convNEAT} verwendet deshalb eine stark modifizierte Variante des \textit{k-Medoid} Algorithmuses.
				Im Gegensatz zum Mittelpunkt wählen wir den \textit{"Median"}, in unserem Fall das Genom im Cluster, dass den geringsten Abstand zu allen anderen
				Clustermitgliedern hat, also $$fit(z) \coloneqq \sum _{x_j \in S_i} dist(x_j, z)$$ minimiert.

				\textit{k-Medoid} kann deshalb so umgebaut werden, dass es nur von $dist$ abhängt.
				Einige Dinge sind noch zu beachten, z.B. darf kein Cluster zu klein werden, da sonst keine sinnvolle Selektion stattfinden kann.
				Damit man messen kann wie sich die einzelnen Netze eines Clusters über die Zeit entwickeln muss zudem eine Art Konsistenz erhalten bleiben.
				Ist das Beste Cluster $S_k$ für ein festes $k$ dann sollte nach dem erneuten Clustern wieder viele Netze $x_j \in S_k$ im neuen Cluster $S'_k$ sein.
				Diese Konsistenz ist bei \textit{k-Medoid} nicht gegeben. Die Güte des Ergebnis im Bezug auf $J$ hängt stark von den Start-Clustermittelpunkten $\mu_i^0$
				ab. \textit{k-Medoid} wird deshalb üblicherweise mit verschiedenen kompliziert gewählten $\mu_i^0$ mehrmals ausgeführt.
				\textit{convNEAT} verzichtet auf eine komplizierte Initialiserung und verwendet stattdessen den Median der schon aus der letzten Generationen
				bekannten Cluster als Startwert. Die Einbüße in der Funktion $J$ können durch die enstehende Konsistenz gerechtfertigt werden.

				Der entwickelte \textit{consistent bounded k-Medoid} kann aber nur für festes $k$ eine Clusterung finden.
				Für \textit{convNEAT} ist es jedoch wichtig die Anzahl der Cluster varieren zu lassen, falls durch Mutation und Crossover viele
				neuartige Netze entstanden sind oder nach der Selektion zu wenige Netze eines Cluster übrig bleiben.

				\textit{convNEAT} ändert adaptiv die Anzahl der Cluster, falls dies nötig ist und achtet trotzdem darauf, dass die Konsistenz der Cluster erhalten bleibt.

				% Abbild Distanzmatrix

				% Abbild Clusterung von Netzen


			



	
	

\clearpage

\printbibliography[heading=bibintoc, title={References}]

\clearpage

\section*{Declaration}
	\vspace{2cm}
	
	\large
	I, Mark-Oliver Wolf, avouch that I have created this thesis on my own and without help or sources but those explicitly mentioned and that I have marked any and all citations as such.
	
	\vspace{3cm}
	\begin{figure}[b]
		Kaiserslautern, \today
		
		\vspace{2em}
		\begin{tabularx}{\linewidth}{@{}XX@{}}
			\hrule&\\
			Mark-Oliver Wolf & \\
		\end{tabularx}
	\end{figure}

\end{document}
